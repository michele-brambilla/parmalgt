\documentclass[a4paper]{scrartcl}

\usepackage{amsmath}

\title{Links with an Abelian background field.}

\newcommand{\ord}[1]{\ensuremath{^{(#1)}}}

\begin{document}
\maketitle

Assume we have a background field $V$ and fluctuations $q$, such that
we can write the links as
%
\begin{equation}
  \label{eq:1}
  U = \exp\{q\} V.
\end{equation}
%
We will use two different ways to express the perturbative expansion
of (\ref{eq:1}),
%
\begin{align}
  \label{eq:3}
  U =&\; V + g\, U\ord 1 + g^2\, U\ord 2 + \ldots\\
  \label{eq:5}
    =&\; \left (
      1 + g\, \tilde U\ord 1 + g^2\, \tilde U\ord 2 + \ldots
      \right) V \equiv \tilde U \,V.
\end{align}
%
The connection to the perturbative expansion of the algebra
valued $q$-field,
%
\begin{equation}
  \label{eq:2}
  q = g\, q\ord 1 + g^2 q \ord 2 + \ldots
\end{equation}
%
can be established writing (\ref{eq:1}) explicitly,
%
\begin{equation}
  \label{eq:4}
  \exp \{ q \} = 1 + g \, q \ord 1 + g^2\, \left\{
    q \ord 2 + \frac 1 2 \left(q\ord 1\right)^2 \right\}
   + \ldots,
\end{equation}
%
and identifying the terms in the series with the ones in
(\ref{eq:5}) order by order. To go from the group to the algebra, one
has to calculate
%
\begin{equation}
  \label{eq:6}
  q = \log\{\tilde U\} = g\, \tilde U \ord 1 + g^2 \, \left\{
    \tilde U \ord 2 - \frac 1 2 \left( \tilde U \ord 1 \right)^2
    \right \} + \ldots,
\end{equation}
%
and compare the terms with the naive series
%
\begin{equation}
  \label{eq:7}
  q = g\,q\ord 1 + g^2 \,q\ord 2 + \ldots
\end{equation}
%
again, order by order in the coupling.

\subsection*{Arithmetic}
\label{sec:arithmetic}

The multiplication rule for the representation (\ref{eq:3}) is rather
simple, we get (setting $U\ord 0 \equiv V$ and $\tilde U \ord 0 \equiv
1$)
%
\begin{equation}
  \label{eq:8}
  \left ( U' \times U \right)\ord i = \sum_{j = 0}^{i} {U'}\ord j U
  \ord{i-j} = \sum_{j = 0}^{i} {\tilde U '}\hspace{0em}\ord j V' 
  {\tilde U} \ord {i- j}V.
\end{equation}
%
Finally, we have
%
\begin{equation}
  \label{eq:9}
  \widetilde {\left( U' \times U \right)}\ord i = \left ( U' \times U
  \right)\ord i \left(V'\times V \right)^{-1}.
\end{equation}
%
For the addition, we have the following rules,
%
\begin{equation}
  \label{eq:10}
  \left ( U' + U \right)\ord i = {U'}\ord i + U \ord i = {\tilde U'}
  \hspace{0em}\ord i V' + {\tilde U}\ord i V,
\end{equation}
%
and again, we must perform
%
\begin{equation}
  \label{eq:11}
  \widetilde {\left( U' + U\right)}\ord i = \left ( U' + U
  \right)\ord i \left(V' + V \right)^{-1}.
\end{equation}

\subsection*{Implementation}
\label{sec:implementation}

Inspecting the addition and multiplication rules given in the last
section, it seems beneficial to store the link variables in the
expansion (\ref{eq:3}). Only for applying the exponential function,
the form (\ref{eq:5}) is needed.

\subsubsection*{Exponential and Logarithm}
\label{sec:expon-logar}

If the natural logarithm $\log$ is applied to a \texttt{BGptSU3}
instance, the method will assume that the link is stored as described
above and first calculate
%
\begin{equation}
  \label{eq:12}
  \tilde U = 1 + g\,U\ord 1 V^{-1} + g^2\,U \ord 2 V^{-1} + \ldots.
\end{equation}
%
Then, it will calculate $q = \log \{\tilde U\}$ using the Mercator
series
%
\begin{equation}
  \label{eq:13}
  \log \{\tilde U\} = \sum_{i=1}^{\mathrm{PT\_ORD}} 
  (-1)^{n+1} \frac {(\tilde U - 1)^n} n,
\end{equation}
%
aborting the series at the highest perturbative order considered
(i.e. \texttt{PT\_ORD} in the code).

When the exponential function is applied, the code assumes that we
have stored the algebra valued fields $q$ in the \texttt{BGptSU3}
instance given as the argument to the $\exp$ function. The background
field stored in any \texttt{BGptSU3} is \emph{always} assumed to be
$V$. The exponential is then evaluated in the usual way,
%
\begin{equation}
  \label{eq:14}
  \tilde U = \exp \{ q\} = \sum_{i = 0}^{\mathrm{PT\_ORD}} \frac {q^n}{n!},
\end{equation}
%
and finally the $U\ord i = {\tilde U}\ord i V$ are returned. One may
now check that indeed $U = \exp\{\log\{U\}\}V$ holds.
\end{document}
